\documentclass[11pt, oneside]{article}   	% use "amsart" instead of "article" for AMSLaTeX format
\usepackage{geometry}                		% See geometry.pdf to learn the layout options. There are lots.
\geometry{a4paper}                   		% ... or a4paper or a5paper or ... 
%\geometry{landscape}                		% Activate for rotated page geometry
%\usepackage[parfill]{parskip}    		% Activate to begin paragraphs with an empty line rather than an indent
\usepackage{graphicx}				% Use pdf, png, jpg, or eps§ with pdflatex; use eps in DVI mode
								% TeX will automatically convert eps --> pdf in pdflatex		
\usepackage{amssymb}

\usepackage{siunitx}

%SetFonts

%SetFonts


\title{Determination of Residual Stress by Neutron Diffraction Analysis}
\author{Felix Gifford\\c3260374}
%\date{}							% Activate to display a given date or no date

\begin{document}

\maketitle
\section{Abstract}
\section{Introduction and Background}
\subsection{Diffraction}
\section{Experimental Details}
\subsection{Testing of Mechanical Properties}
In order to determine the residual stress in a material by the neutron diffraction method, it is critical to know some of the important mechanical properties of the material. Namely: the Young's Modulus ($E$), and Poisson's Ratio ($\nu$). The Young's Modulus (or Modulus of Elasticity) is the relationship between stress ($\sigma$) and strain ($\epsilon$).
$$\sigma = E\epsilon$$
Poisson's Ratio is used for stress and strain in three dimensions:

The mechanical properties of the materials were determinedby the use of the University's tensile testing facilities on small samples of each material. The samples used were small 'dogbone' coupons, cut from the original stock plates of aluminium. A schematic of these 'dogbones' is shown below.

These coupons wer put through a standard tensile test, involving the application of an increasing tensile load until failure while simultaneously recording measurements with a load cell and a strain gauge. Once the sample had passed passed it's yield stress and begun plastic deformation the strain gauge was remooved the and load was increased to the point of failure. The final load ($F_F$), and ultimate strain ($\sigma_U$) are recorded in \ref{tab:a}.
\begin{table}
	\centering
	\caption{Final load and ultimate strain measured for the tensile test coupons}\label{tab:a}
	\begin{tabular}[c]{c | c c}
	Sample & $F_F$ (\si{\mega\pascal}) & $\sigma_U$ \\ \hline\
	6061-T6 & 18.1 & 0.2 \\
	7075-T6 & 34.7 & 0.18 \\
	\end{tabular}
\end{table}
\section{Results}
\section{Discussion}
\section{Conclusion}
\section{References}

\end{document} 